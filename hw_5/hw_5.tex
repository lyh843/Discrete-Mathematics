\documentclass{article}
\usepackage{amsthm}
\usepackage{graphicx}
\usepackage{ctex}
\usepackage{amsmath}
\usepackage{amssymb} % 添加 amssymb 以支持更多符号
\usepackage{amsfonts}
\usepackage{tikz}
\usepackage{cancel}
\usepackage{listings}
\usetikzlibrary{arrows.meta} % 箭头样式
\title{离散数学作业\_5}
\author{李云浩 241880324}
\date{\today}
\begin{document}
\maketitle
\section{5.1}
\subsection{5}
每一个整数都有其对应的唯一平方数,且整数的平方数一定为整数。
因此$dom\ f = A \land ran\ f \subseteq B \land \forall x \in A, f(x)$的值唯一。
满足函数的要求。
\subsection{6}
对每一个实数$a$,都有唯一对应的$e^a$,并且$\forall a \in R \rightarrow e^a \in R$。
因此$dom\ f = A \land ran\ f \subseteq B \land \forall x \in A, f(x)$的值唯一。
满足函数的要求。
\subsection{7}
对于每一个实数,要么是整数、要么不是整数,并且不能同时即是整数又不是整数。
因此$dom\ f = A \land (\forall x \in A, f(x) = 1 \lor f(x) = 0)$。满足函数的要求。
\subsection{8}
对于每一个实数,都有唯一对应的小于或等于该实数的最大整数。
因此$dom\ f = A \land ran\ f \subseteq B \land \forall x \in A, f(x)$的值唯一。
满足函数的要求。
\subsection{11}
(a) $\forall b (b\in B \rightarrow (\exists a \in A \land afb))$,因此满足满射。\\
$\forall a_1 \in A \forall a_2 \in A(a_1 \neq a_2 \rightarrow f(a_1) \neq f(a_2))$,因此满足单射。\\
(b) $\nexists a \in A$使得$f(a) = b \lor f(s) = d$,因此不满足满射。\\
$1 \neq 2 \land f(1) = f(2) = a$,因此不满足单射。\\
\subsection{12}
(a) $\nexists a \in A$使得$f(a) = z$,因此不满足满射。\\
$\forall a_1 \in A \forall a_2 \in A(a_1 \neq a_2 \rightarrow f(a_1) \neq f(a_2))$,因此满足单射。\\
(b) $\forall b \in B, \exists a \in A \rightarrow afb$,因此满足满射。\\
$1.1 \neq 0.06 \land f(1.1) = f(0.06) = p$,因此不满足单射。\\
\subsection{13}
(a) $\forall b \in B$都有$(b + 1) \in A \rightarrow f(b + 1) = b$,因此满足满射。\\
$\forall a_1 \in A, \forall a_2 \in A(a_1 \neq a_2 \rightarrow a_1 - 1 \neq a_2 - 1 )$,因此满足单射。\\
(b) $\forall b \in B$都有$b \in A \land -b \in A \rightarrow f(b) = b \land f(-b) = b$,因此满足满射。\\
$-1 \in A \land 1 \in A \land -1 \neq 1 \rightarrow f(-1) = f(1)$,因此不满足单射。
\subsection{14}
(a) $\forall b \in B(\exists (b, 1) \in A \rightarrow f((b, 1)) = b)$,因此满足满射。\\
$\exists (a, 1) \in A(\exists (a, 2) \in A \rightarrow f((a, 1)) = f((a, 2)) = a)$,因此不满足单射。\\
(b) 对于$B$中的元素$(2, a)$,显然$A$中没有任何一个元素能通过函数关系指向它,因此不满足满射。\\
$\exists (1,b) \in A, \exists (2, b) \in A \rightarrow f((1, b)) = f((2, b)) = (1, a)$,因此不满足单射。
\subsection{15}
(a) $\forall (m, n) \in B(\exists (\frac{m + n}{2}, \frac{m - n}{2}) \in A \rightarrow f((\frac{m + n}{2}, \frac{m - n}{2}))
= (m, n))$,因此满足满射。\\
$\forall (m, n) \in B$,有且仅有$(\frac{m + n}{2}, \frac{m - n}{2}) \in A$,使得$f(\frac{m + n}{2}, \frac{m - n}{2}) = (m, n)$。因此满足单射。\\
(b) $\forall b \in B, (\exists \sqrt{b} \in A \rightarrow f( \sqrt{b} ) = b)$,因此满足满射。\\
$\exists a \in A \exists -a \in A(a \notin -a \rightarrow f(a) = f(-a) = a^2)$,因此不满足单射。
\subsection{29}
对于每个$A$中的元素,其都可以通过函数指向$B$中的任一元素,因此函数的个数为$n^m$个。
\subsection{30}
$g:B \rightarrow C$为单射函数\\
$f:A \rightarrow B$为双射函数
\subsection{31}
$C$, $B$, 因为$g:B \rightarrow C$是满射的,$A$,因为$f:A \rightarrow B$是满射的。
\subsection{33}
因为$g \circ f$是满射的,则对于$\forall c \in C (\exists f(x) \rightarrow g(f(x)) = c)$。
因为$f(x) \in B$,因此$g:B \rightarrow C$是满射的。
\subsection{34}
因为$O(a_1, f) \cap O(a_2, f) \neq \varnothing$,所以$\exists f^{m_1}(a_1) = f^{m_2}(a_2)$。因此$\forall n \in Z,
(f^n(a_1) = f^{n - m_1 + m_2}(a_2))$,因此$O(a_1, f) = O(a_2, f)$。
\subsection{40}
(a) 令$a_1 = -1, a_2 = 1$,此时$f(a_1 + a_2) = f(0) = 0$,$f(a_1) = 1, f(a_2) = 1 \rightarrow f(a_1) + f(a_2) = 2$。
因此$f(a_1 + a_2) \neq f(a_1) + f(a_2)$,因此原式不成立。\\
(b) 因为$s_1 \cdot s_2$即两个字符串进行拼接,长度为原来两条字符串各自长度的加和。故$f(s_1 \cdot s_2) = f(s_1) + f(s_2)$
显然成立。
\subsection{41}
(a)因为$a, b \in A \land A = \{0, 1\}$,并且$a \diamond b = (a + b)mod 2$。因此当且仅当$a = b$时,
$(a \diamond b) = 0$。反例:$f(1 \diamond 1) = f(0)$为假,$f(1) \lor f(1) = f(1)$为真。因此$f(1 \diamond 1) \neq f(1) \lor f(1)$。\\
(b)$$
\begin{array}{|c|c|c|c|c|c|c|}
    \hline
    a & b & a \diamond b & f(a) & f(b) & f(a \diamond b) & f(a) \land f(b)\\
    \hline
    0 & 0 & 0 & 0 & 0 & \text{假} & \text{假}\\
    \hline
    0 & 1 & 1 & 0 & 1 & \text{真} & \text{假}\\
    \hline
    1 & 0 & 1 & 1 & 0 & \text{真} & \text{假}\\
    \hline
    1 & 1 & 0 & 1 & 1 & \text{假} & \text{真}\\
    \hline
\end{array}
$$因此$f(a \diamond b) \neq f(a) \land f(b)$
\section{5.2}
\subsection{7}
不妨设$n = pk + q$,其中$(0 \leq q < k)$。因此从$1 \sim n$中,一共有$p$个$k$的倍数。且$\lfloor \frac{n}{k} \rfloor = p$。
因此在$1 \sim n$之间,$k$的倍数的个数是$\lfloor \frac{n}{k} \rfloor$个。
\subsection{8}
因为$n$是奇数,所以$n = 2k - 1, k \in Z$。因此$n^2 =  (2k - 1)^2 = 4k^2 - 4k + 1$。因此$\lceil \frac{n^2}{4} \rceil = 
k^2 - k + 1$,且$\frac{n^2 + 3}{4} = \frac{4k^k - 4k + 4}{4} = k^2 - k + 1$。因此$\lceil \frac{n^2}{4} \rceil = \frac{n^2 + 3}{4}$。
\subsection{18}
(a) 对于每一个$A^*$中的字符串,它都一定有确切的长度。因此$l$是处处有定义的。\\
(b) 反例:对于$a, b$,$l(a) = l(b) = 1$,因此$l$不是单射的。\\
(c) 题目中给出的函数的陪域是所有整数,但$A^*$中不存在长度为负数的字符串,因此$l$不是满射的。
\subsection{20}
对于一个布尔变量时,变量的值有两种情况。对于每个变量的值,其对应的布尔函数的值也有两种情况,因此总共有4个不同的关于p的布尔函数。\\
对于两个布尔变量时,变量的值有$2 \times 2 = 4$种情况。对于每个变量的值,其对应的布尔函数的值有两种情况,因此总共有$2^4 = 16$个不同的
有两个布尔变量的布尔函数。
\subsection{28}
根据特征函数的定义:$$
f(x) = 
\begin{cases}
    1, & x \in A\\
    0, & x \notin A
\end{cases}
$$
因为$A$中共有$n$个元素,对于$A$的任意一个子集,即将$A$中部分元素舍弃掉,此时该元素在子集中的特征方程取值为0。
一共有$n$个元素,因此能组成$2^n$个不同的01串,即表示不同的子集。因此$|pow(A)| = 2^n$。
\subsection{29}
因为$f_A$是$A$关于全集$U$的特征函数,因此有$$
\forall x \in U, f(x) = 
\begin{cases}
    1, & x \in A\\
    0, & x \notin A
\end{cases}
$$
因此$f^{-1}$表示:$$
\begin{cases}
    x \in A, & 1\\
    x \notin A, & 0
\end{cases}
$$
因此$f^{-1}(1)$表示所有在集合$A$中的元素。
\section{5.3}
\subsection{11}
$\Theta(n\lg(n)):f_1, f_{12} \quad \Theta(n^2):f_2 \quad \Theta(a^n):f_3 \quad \Theta(lg(n)):f_4 \quad \Theta(1):f_5\\
\Theta(n):f_6, f_{10}, f_{11} \quad \Theta(lg(lg(n))): f_7 \quad \Theta(n^{0.7}): f_8 \quad \Theta(n^n):f_9$
\subsection{12}
$\Theta(1) < \Theta(\lg(\lg(n))) < \Theta(\lg(n)) < \Theta(n\lg(n)) < \Theta(n^{0.7}) < \Theta(n) < \Theta(n^2)
< \Theta(a^n) < \Theta(n^n)$
\subsection{13}
$\Theta(1) : f_5 \quad \Theta(n) : f_6, f_{10}, f_{11} \quad \Theta(n\lg(n)) : f_1 \quad \Theta(\lg(n)) : f_4\\$
$\Theta(n^2) : f_2 \quad \Theta(\sqrt{n}): \text{无} \quad \Theta(2^n) : \text{无}$
\subsection{20}
$f(n) = 2n \in \Theta(n)$.
\subsection{21}
$if$判断中一步,$N$和$Q$的两层$for$循环中有一步,三层$N, Q, M$的循环中,要执行乘法、加法、赋值的三步操作,
最后还有一步$return$操作。因此步骤函数为$f(n) = 1 + NQ + 3NQM + 1$。该函数为$\Theta(n^3)$.
\subsection{22}
因为$F(N + 1) = 2F(N) + F(N + 1), F(0), F(1)$的通项公式为$F(n) = \frac{1}{3}2^n - \frac{1}{3}(-1)^n$。
因此$F(N) \in \Theta(2^n)$,并且可以得出$F(N)$所需的运算时间大概为$\frac{1}{3}2^n - \frac{1}{3}(-1)^n$倍的运算$F(0)$的时间。
为$\Theta(2^n)$
\subsection{23}
$(a) P_n = P_{n - 1} + (n - 2) + (n - 3), P_3 = 1, P_4 = 4$\\
(b) 由$(a)$可得递推关系为:$P_n = P_3 + \sum_{3}^{n}(2k - 5) = P_3 + (n + 3)(n - 2) - 5n + 10 = n^2 - 4n + 5$。因此
运行时间为$n^2 - 4n + 5$倍的$P_3$的运行时间。为$\Theta(n^2)$
\subsection{24}
(1) 充分性:$0 < a < b \Rightarrow \Theta(n^a)$比$\Theta(n^b)$低。\\
因为$0 < a < b \rightarrow \frac{n^b}{n^a} = n^{b - a} > 1$,因此$n^b > n^a$。故$\Theta(n^a)$比$\Theta(n^b)$低.
(2) 必要性:$\Theta(n^a)$比$\Theta(n^b)$低$\Rightarrow 0 < a < b$。\\
反证法:当$a = b$时,$\Theta(n^a)$显然与$\Theta(n^b)$同阶。
当$a > b$时,$\lim_{n \rightarrow \infty} \frac{n^b}{n^a} = 0$,与$\Theta{n^a}$比$\Theta{n^b}$低不符。
因此$0 < a < b$。\\
综上所述:$\Theta(n^a)$比$\Theta(n^b)$低当且仅当$0 < a < b$。
\subsection{25}
(1) 充分性:$0 < a < b \Rightarrow \Theta(a^n)$比$\Theta(b^n)$低。\\
因为$0 < a < b \rightarrow a \times a \times \dots \times a < b \times b \times \dots \times b$,
故$\Theta(a^n)$比$\Theta(b^n)$低。
(2) 必要性:$\Theta(a^n)$比$\Theta(b^n)$低$\Rightarrow 0 < a < b$。\\
反正法:当$a = b$,显然$\Theta(a^n)$与$\Theta(b^n)$同阶。
当$a > b$时,由(1)可证得:$\Theta(a^n)$比$\Theta(b^n)$高阶,因此与前提矛盾。
故$0 < a < b$.\\
综上所述:$\Theta(a^n)$比$\Theta(b^n)$低,当且仅当$0 < a < b$。
\subsection{26}
当$r \neq 0$时,存在$c > r \rightarrow |rf(x)| < c|f(x)|$,因此$rf(x)$是$O(f(x))$。
又存在$d > \frac{2}{|r|} \rightarrow d|rf(x)| = 2|f(x)| > |f(x)|$,因此$f(x)$是$O(rf(x))$。
综上所述,$\Theta(rf) = \Theta(f)$。
\subsection{27}
当$\Theta(f)$比$\Theta(g)$低时,有$\exists c \in \mathbf{R} \rightarrow |f(x)| < c|g(x)|$。因为$h(x)$是一个非零函数,
因此$|h(x)| \neq 0 \rightarrow |h(x)||f(x)| < c|h(x)||g(x)|$因此:$\exists c \in \mathbf{R} \rightarrow |h(x)f(x)| < c|h(x)g(x)|$。
即$\Theta(fh)$比$\Theta(gh)$低。
\subsection{28}
因为$\Theta(f) = \Theta(h)$,所以$\exists a_1 \in \mathbf{R} a_1|f| < |h|$。
同理,因为$\Theta(g) = \Theta(h)$,所以$\exists b_1 \in \mathbf{R} b_1 |g| < |h|$。
因此$\frac{a_1 + b_1}{2}(|f| + |g|) < |h||$,所以$f + g$是$O(h)$的。
\subsection{29}
因为$\Theta(f) = \Theta(g)$,所以$\exists a_1, a_2 \in \mathbf{R} a_1|f| < |g|, a_2|f| > |g|$。
对于$\forall c \neq 0$,取$b_1 = \frac{a_1}{|c|}, b_2 = \frac{a_2}{|c|}$,因此$b_1|cf| = a_1|f| < |g| \land
b_2|cf| = a_2|f| > |g|$。因此$\Theta(cf) = \Theta(g)$。
\section{5.4}
\subsection{12}
(a) $(1, 4, 5) \circ (2, 3) \circ (6, 8)$\\
(b) $(1, 2, 3, 4) \circ (5, 7, 8, 6)$
\subsection{13}
(a) $(1, 6, 3, 7, 2, 5, 4, 8)$\\
(b) $(1, 2, 3) \circ (5, 6, 7, 8)$
\subsection{14}
(a) $(a, g, e, c, b, d)$\\
(b) $(a, d, b, e, g, c)$\\
\subsection{15}
(a) $(2, 6) \circ (2, 8) \circ (2, 5) \circ (2, 4) \circ (2, 1)$\\
(b) $(3, 6) \circ (3, 1) \circ (4, 5) \circ (4, 2) \circ (4, 8)$
\subsection{16}
$$
\begin{array}{|c|c|c|c|c|c|c|c|c|c|c|}
    \hline
    1 & 2 & 3 & 4 & 5 & 6 & 7 & 8 & 9 & 10 & 11\\
    \hline
    W & A & E & Y & R & R & H & O & E & U & E\\
    \hline
\end{array}
$$
\subsection{20}
(a) $(1, 4, 6, 8, 3) = (1, 3) \circ (1, 8) \circ (1, 6) \circ (1, 4)$,偶置换\\
(b) $(1, 7, 6, 8, 5) \circ (2, 3, 4) = (1, 5) \circ (1, 8) \circ (1, 6) \circ (1, 7) \circ (2, 4) \circ (2, 3)$,偶置换。 
\subsection{26}
设置换$p$是将$A$中的$b_1, b_2, \dots, b_r$进行循环,则有$\{(b_1, b_2), (b_2, b_3), \dots (b_{r - 1}, b_r)\}$。
因此$p \circ p$为$\{(b_1, b_3), (b_2, b_4), \dots\}$。\\
当$r$为奇数时,$p \circ p$表示循环$(b_1, b_3, \dots b_r, b_2, b_4, \dots b_{r - 1})$。\\
当$r$为偶数时,$p \circ p$可以拆分为两个循环,分别为$(b_1, b_3, \dots b_{r - 1})$,$(b_2, b_4, \dots b_r)$。
\subsection{28}
(a) $p = (1, 4) \circ (2, 3, 5)$\\
(b) $p^{-1} = (4, 1) \circ (2, 5, 3)$\\
(c) $p^2 = (2, 5, 3)$\\
(d) 因为$(1, 4)$的长度为$2$,$(2, 3, 5)$的长度为$3$。因此$k = LCM(2, 3) = 6$
\subsection{29}
(a) 当$n = 1$,已由题目给出,$p$是$A$的一个置换。\\
假设$n = k$时,$p^k$是$A$的一个置换,下证:$n = k + 1$时,$p^{k + 1}$是$A$的置换。\\
因为$p^k$是$A$上的一个置换,因此$p^k$是从$A$到$A$的双射函数。同理,$p$也是$A$上的一个双射函数。
因为双射函数复合双射函数仍是双射函数,因此$p^k \circ p = p^{k + 1}$仍是$A$上的一个双射函数,
故$p^{k + 1}$是$A$的置换。\\
综上所述,如果$p$是$A$的一个置换,那么$p^n$也是$A$的一个置换。\\
(b) 对于一个长度为$r$的循环$a = \{(a_1, a_2, \dots, a_r)\}$,$a^r$表示其中长度为$r$的路径,
该路径长度恰好为循环的长度,因此所有元素都回到自身,即$a^r = I_a$。\\
不妨将$p$化为多个互不相交的循环的复合形式,长度分别为$n_1, n_2, \dots n_k$。由上面的结论可知,
当$m = LCM(n_1, n_2, \dots, n_k)$时,满足所有元素都回到自身,因此$p^m = I_A$。
\subsection{30}
不妨设$p$能够转化为多个循环的复合形式,其中一个循环为:$(c_1, c_2, \dots, c_r)$。下证$R$是一个等价关系:\\
(1) 自反性:对于$\forall x \in A, p^0(x) = x$。因此$\forall x \in A \rightarrow x R x$。满足自反性。\\
(2) 对称性:对于$\forall (a, b) \in R \land a \neq b$,因此$a, b$一定是$c_1 \sim c_r$中的一个元素。设$a = c_i, b = c_j\ (i < j)$。
则:$p^{j - i}(a) = b \land p^{r + i - j}(b) = a$。因此$(b, a) \in R$。所以$\forall (a, b) \in R \rightarrow (b, a) \in R$,满足对称性。\\
(3) 传递性:对于$\forall (a, b), (b, d) \in R$,当$ a = b \lor b = d$时显然满足传递性。当$a \neq b \land b \neq d$时,$a, b, d$一定是$c_1, c_2, \dots, c_r$
中的元素,因此不妨设$a = c_i, b = c_j, d = c_k\ (i < j < k)$,因此有$p^{j - i}(a) = b \land p^{k - j}(b) = d \rightarrow p^{k - i}(a) = d$。
故$\forall (a, b), (b, d) \in R \rightarrow (a, d) \in R$。满足传递性。\\
因此$R$是一个等价关系。\\
将$p$化为多个互不相交的循环的复合形式。每个循环中的元素组成$R$的一个等价类。剩下单独的元素分别独自组成$R$的等价类。
\subsection{37}
(a) $3$个,分别为$(1, 2, 3), (1, 3, 2), (2, 1, 3)$\\
(b) 任选两个摆在奇数位上:$C_4^2 = 6$个,其余都能确定。因此总共$6$个。
\subsection{38}
任意选择两个数字摆在奇数位上:$C_5^2 = 10$个,由于具有递增递减性质,因此其余元素的位置也能确定。因此总共有$10$个。
\subsection{39}
只要选出什么数字摆在奇数位上,则该序列便能确定。当$n$为偶数时,奇数位有$\frac{n}{2}$个数字。当$n$为基数时,奇数位有$\frac{n + 1}{2}$个数字。
因此,奇数位的数字有$\lceil \frac{n}{2} \rceil$个。故$A$的增-减置换数等于由$A$中元素形成的长度为$\lceil \frac{n}{2} \rceil$的递增序列的数目。
\end{document}