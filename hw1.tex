\documentclass{article}
\usepackage{amsthm}
\usepackage{graphicx}
\usepackage{ctex}
\usepackage{amsmath}
\usepackage{amssymb} % 添加 amssymb 以支持更多符号
\usepackage{amsfonts}
% \usepackage{bussproofs}
\title{离散数学作业\_1}
\author{李云浩 241880324}
\date{\today}
\begin{document}
\maketitle
\section{\textbf{P60-62}}
\textbf{T12}

\textit{p} 表示今天是星期一, \textit{q} 表示碟子里有汤匙,\textit{r} 表示草地是湿的。

(a) $\textit{p} \land \textit{q}$ \quad (b) $\textit{q} \lor \textit{r}$ \quad (c) $\lnot \textit{p} \land \lnot \textit{r}$ \quad (d) $\lnot \textit{q} \land \textit{r}$ 

\vspace{10pt}

\textbf{T15}

(a) 对于任意的整数,都存在一个整数,使它们之和为偶数。

(b) 存在一个整数,使得它与任意的整数之和都为偶数。

\vspace{10pt}

\textbf{T16}

(a) 所有的整数都不是素数 \quad (b) 存在一个不是偶数的整数。

\vspace{10pt}

\textbf{T18}

(a) $\forall x \lnot P(x)$ \quad (b) $\forall x \forall y R(x, y)$

(c) $\lnot(\exists x (P(x) \land Q(x)))$ \quad (d) $\forall x (P(x) \lor Q(x))$

\vspace{10pt}

\textbf{T28}

\[
\begin{array}{|c|c|c|c|c|}
    \hline
    p & q & p \lor q & \lnot q & (p \lor q) \lor \lnot q\\
    \hline
    F & F & F & T & T\\
    F & T & T & F & T\\
    T & F & T & T & T\\
    T & T & T & F & T\\
    \hline
\end{array}
\]

\vspace{10pt}

\textbf{T30}

\[
\begin{array}{|c|c|c|c|c|c|c|}
    \hline
    p & q & r & \lnot p & \lnot r & \lnot p \lor q & (\lnot p \lor q) \land \lnot r\\
    \hline
    F & F & F & T & T & T & T\\
    F & F & T & T & F & T & F\\
    F & T & F & T & T & T & T\\
    F & T & T & T & F & T & F\\
    T & F & F & F & T & F & F\\
    T & F & T & F & F & F & F\\
    T & T & F & F & T & T & T\\
    T & T & T & F & F & T & F\\
    \hline
\end{array}
\]

\vspace{10pt}

\textbf{T32}

\[
\begin{array}{|c|c|c|c|c|c|}
    \hline
    p & q & r & p \downarrow q & p \downarrow r & (p \downarrow q) \land (p \downarrow r)\\
    \hline
    F & F & F & T & T & T\\
    F & F & T & T & F & F\\
    F & T & F & F & T & F\\
    F & T & T & F & F & F\\
    T & F & F & F & F & F\\
    T & F & T & F & F & F\\
    T & T & F & F & F & F\\
    T & T & T & F & F & F\\
    \hline
\end{array}
\]

\vspace{10pt}

\textbf{T35}

\[
\begin{array}{|c|c|c|c|}
    \hline
    p & q & p \land q & (p \land q) \triangle p\\
    \hline
    F & F & F & F\\
    F & T & F & F\\
    T & F & F & T\\
    T & T & T & F\\
    \hline
\end{array}
\]

\section{\textbf{P66-69}}

\textbf{T6}

(a)$\lnot r \rightarrow q$ \quad (b) $\lnot q \land p$ \quad (c) $q \rightarrow \lnot p$ \quad (d) $\lnot p \rightarrow \lnot r$

\vspace{10pt}

\textbf{T7}

(a) 仅当我心情不好时,我会去看电影并且不学习离散数学结构。

(b) 如果我心情好,那么我会学习离散数学结构或者去看电影。

(c) 如果我心情不好,那么我不会去看电影或者我会学习离散数学结构。

(d) 当且仅当我心情很好,我会去看电影并且不学习离散数学结构。

\vspace{10pt}

\textbf{T12}

(a) 不定式
\[
\begin{array}{|c|c|c|c|c|c|}
    \hline
    p & q & q \land p & \lnot p & q \land \lnot p & (q \land p) \lor (q \land \lnot p)\\
    \hline
    F & F & F & T & F & F\\
    F & T & F & T & T & T\\
    T & F & F & F & F & F\\
    T & T & T & F & F & T\\
    \hline
\end{array}
\]

(b) 重言式
\[
\begin{array}{|c|c|c|c|}
    \hline
    p & q & p \land q & (p \land q) \rightarrow p\\
    \hline
    F & F & F & T\\
    F & T & F & T\\
    T & F & F & T\\
    T & T & T & T\\
    \hline    
\end{array}
\]

(c) 不定式
\[
\begin{array}{|c|c|c|c|}
    \hline
    p & q & q \land p & p \rightarrow (q \land p)\\
    \hline
    F & F & F & T\\
    F & T & F & T\\
    T & F & F & F\\
    T & T & T & T\\
    \hline
\end{array}
\]

\vspace{10pt}

\textbf{T13}

$p \rightarrow q$ 的真值表为:\[
\begin{array}{|c|c|c|}
    \hline
    p & q & p \rightarrow q\\
    \hline
    F & F & T\\
    F & T & T\\
    T & F & F\\
    T & T & T\\
    \hline    
\end{array}
\]

因此 $p \rightarrow q$为假,即$p = T, q = F$,此时:
\[
\begin{array}{|c|c|c|c|c|}
    \hline
    p & q & (p \land q) & \lnot (p \land q) & (\lnot (p \land q)) \rightarrow q\\
    \hline
    T & F & F & T & F\\
    \hline
\end{array}
\]

$(\lnot (p \land q)) \rightarrow q$真值为假。

\vspace{10pt}

\textbf{T14}

$p \rightarrow q$ 的真值表为:\[
\begin{array}{|c|c|c|}
    \hline
    p & q & p \rightarrow q\\
    \hline
    F & F & T\\
    F & T & T\\
    T & F & F\\
    T & T & T\\
    \hline    
\end{array}
\]

因此 $p \rightarrow q$为假,即$p = T, q = F$,此时:
\[
\begin{array}{|c|c|c|c|c|}
    \hline
    p & q & \lnot p & (p \rightarrow q) & (\lnot p) \lor (p \rightarrow q)\\
    \hline
    T & F & F & F & F\\
    \hline
\end{array}
\]

$(\lnot p) \lor (p \rightarrow q)$ 真值为假。

\vspace{10pt}

\textbf{T15}

$p \rightarrow q$ 的真值表为:\[
\begin{array}{|c|c|c|}
    \hline
    p & q & p \rightarrow q\\
    \hline
    F & F & T\\
    F & T & T\\
    T & F & F\\
    T & T & T\\
    \hline    
\end{array}
\]

因此 $p \rightarrow q$为真,即$(p = F) \lor (p = q = T)$,此时:
\[
\begin{array}{|c|c|c|c|c|}
    \hline
    p & q & p \land q & \lnot q & (p \land q) \rightarrow \lnot q\\
    \hline
    F & F & F & T & T\\
    F & T & F & F & T\\
    T & T & T & F & F\\
    \hline
\end{array}
\]

$(p \land q) \rightarrow \lnot q$ 的真值可能为真可能为假,无法确定。

\vspace{10pt}

\textbf{T16}

$p \rightarrow q$ 的真值表为:\[
\begin{array}{|c|c|c|}
    \hline
    p & q & p \rightarrow q\\
    \hline
    F & F & T\\
    F & T & T\\
    T & F & F\\
    T & T & T\\
    \hline    
\end{array}
\]

因此 $p \rightarrow q$为真,即$(p = F) \lor (p = q = T)$,此时:
\[
\begin{array}{|c|c|c|c|c|c|}
    \hline
    p & q & (p \rightarrow q) & \lnot (p \rightarrow q) & \lnot p & \lnot (p \rightarrow q) \land \lnot p\\
    \hline
    F & F & T & F & T & F\\
    F & T & T & F & T & F\\
    T & T & T & F & F & F\\
    \hline
\end{array}
\]

$\lnot (p \rightarrow q) \land \lnot p$的真值为假。

\vspace{10pt}

\textbf{T24}

(a) 天气好或者我去上班。

(b) 卡罗尔没有生病,去了野餐但卡罗尔玩得不愉快。

(c) 我进入了比赛并且赢得了这场比赛。

\vspace{10pt}

\textbf{T27}

$p \land (q \lor r)$ 的真值表为:
\[
\begin{array}{|c|c|c|c|c|}
    \hline
    p & q & r & q \lor r & p \land (q \lor r)\\
    \hline
    F & F & F & F & F\\
    F & F & T & T & F\\
    F & T & F & T & F\\
    F & T & T & T & F\\
    T & F & F & F & F\\
    T & F & T & T & T\\
    T & T & F & T & T\\
    T & T & T & T & T\\
    \hline
\end{array}
\]

$(p \land q) \lor (p \land r)$ 的真值表为:
\[
\begin{array}{|c|c|c|c|c|c|}
    \hline
    p & q & r & p \land q & p \land r & (p \land q) \lor (p \land r)\\
    \hline
    F & F & F & F & F & F\\
    F & F & T & F & F & F\\
    F & T & F & F & F & F\\
    F & T & T & F & F & F\\
    T & F & F & F & F & F\\
    T & F & T & F & T & T\\
    T & T & F & T & F & T\\
    T & T & T & T & T & T\\
    \hline    
\end{array}
\]

因此在$\textit{p q r}$ 的不同的真值指派情况下,两个命题的真值情况一致,故两个命题等价。
即$p \land (q \lor r) \equiv (p \land q) \lor (p \land r)$

\vspace{10pt}

\textbf{T28}

$\lnot (p \lor q)$ 的真值表为:
\[
\begin{array}{|c|c|c|c|}
    \hline
    p & q & p \lor q & \lnot (p \lor q)\\
    \hline
    F & F & F & T\\
    F & T & T & F\\
    T & F & T & F\\
    T & T & T & F\\
    \hline
\end{array}
\]

$(\lnot p) \land (\lnot q)$ 的真值表为:
\[
\begin{array}{|c|c|c|c|c|}
    \hline
    p & q & \lnot p & \lnot q & (\lnot p) \land (\lnot q)\\
    \hline
    F & F & T & T & T\\
    F & T & T & F & F\\
    T & F & F & T & F\\
    T & T & F & F & F\\
    \hline
\end{array}
\]

因此在$\textit{p q}$ 的不同的真值指派情况下,两个命题的真值情况一致,故两个命题等价。
即$\lnot (p \lor q) \equiv (\lnot p) \land (\lnot q)$

\vspace{10pt}

\textbf{T29}
\begin{align*}
    \lnot (p \leftrightarrow q) &\equiv \lnot ((p \land q) \lor (\lnot p \land \lnot q))\\
    &\equiv \lnot (p \land q) \land \lnot (\lnot p \land \lnot q)\\
    &\equiv (\lnot p \lor \lnot q) \land (p \lor q)\\
    &\equiv ((\lnot p \lor \lnot q) \land p) \lor ((\lnot p \lor \lnot q) \land q)\\
    &\equiv ((\lnot p \land p) \lor (p \land \lnot q)) \lor ((\lnot p \land q) \lor (\lnot q \land q))\\
    &\equiv (F \lor (p \land \lnot q)) \lor ((\lnot p \land q) \lor F)\\
    &\equiv ((p \land \lnot q) \lor (q \land \lnot p))\\
\end{align*}

\vspace{10pt}

\textbf{T30}
\[
\begin{array}{cc}
    1. & \exists x (P(x) \lor Q(x))\\
    2. & P(a) \lor Q(b)\\
    \hline
    3. & \therefore \exists x P(x) \lor \exists x Q(x)\\
\end{array}
\]

因此 $\exists x (P(x) \lor Q(x)) \equiv \exists x P(x) \lor \exists x Q(x)$

\vspace{10pt}

\textbf{T31}
\begin{align*}
\forall x (P(x) \land Q(x)) &\equiv \lnot (\exists x (P(x) \lor Q(x)))\\
&\equiv \lnot (\exists x P(x) \lor \exists x Q(x))\\
&\equiv \forall x P(x) \land \forall x Q(x)\\    
\end{align*}

因此 $\forall x(P(x) \land Q(x)) \equiv \forall x P(x) \land \forall x  Q(x)$

\vspace{10pt}

\textbf{T32}
\begin{align*}
    (p \land q) \rightarrow p &\equiv \lnot(p \land q) \lor (p \land q \land p)\\
    &\equiv (\lnot p \lor \lnot q) \lor (p \land q)\\
    &\equiv (\lnot p \lor \lnot q \lor p) \land (\lnot p \lor \lnot q \lor q)\\
    &\equiv T \land T\\
    &\equiv T\\
\end{align*}

\vspace{10pt}

\textbf{T33}
\begin{align*}
    q \rightarrow (p \lor q) &\equiv \lnot q \lor (p \lor q)\\
    &\equiv \lnot q \lor p \lor q\\
    &\equiv T\\
\end{align*}

\vspace{10pt}

\textbf{T34}
\begin{align*}
    (p \land (p \rightarrow q)) \rightarrow q &\equiv (p \land (\lnot p \lor q)) \rightarrow q\\
    &\equiv ((p \land \lnot p) \lor (p \land q)) \rightarrow q\\
    &\equiv \lnot(p \land q) \lor q\\
    &\equiv \lnot p \lor \lnot q \lor q\\
    &\equiv T\\
\end{align*}

\vspace{10pt}

\textbf{T35}
\begin{align*}
    ((p \rightarrow q) \land (q \rightarrow r)) \rightarrow (p \rightarrow r) &\equiv ((\lnot p \lor q) \land (\lnot q \lor r)) \rightarrow (\lnot p \lor r)\\
    &\equiv \lnot ((\lnot p \lor q) \land (\lnot q \lor r)) \lor (\lnot p \lor r)\\
    &\equiv (\lnot (\lnot p \lor q) \lor \lnot (\lnot q \lor r)) \lor (\lnot p \lor r)\\
    &\equiv (p \land \lnot q) \lor (q \land \lnot r) \lor \lnot p \lor r\\
    &\equiv (\lnot q \lor \lnot p) \lor (q \lor r)\\
    &\equiv T\\
\end{align*}

\section{\textbf{P73-75}}

\textbf{T8}

\textit{p、q、r、s} 分别表示这个学期毕业、通过了物理考试、每星期花10个小时学习物理、打排球。

前提:$p \rightarrow q$、$\lnot r \rightarrow \lnot q$、$r \rightarrow \lnot s$、$s$
证明:$\lnot p$

\[
\begin{array}{cc}
    1. & 
\end{array}
\]

\vspace{10pt}

\textbf{T12}

(a)
\[
\begin{array}{cc}
    1. & (p \rightarrow q) \land (q \rightarrow r)\\
    2. & (\lnot p \lor q) \land (\lnot q \lor r)\\
    3. & \lnot q \land r\\
    4. & \lnot q\\
    5. & \lnot p \lor q\\
    6. & \lnot q\\
    \hline
    7. & \therefore p\\
\end{array}
\]

(b)
\[
\begin{array}{cc}
    1. & \lnot (p \rightarrow q)\\
    2. & \lnot (\lnot p \lor q)\\
    3. & p \land \lnot q\\
    4. & p\\
    \hline 
    5. & \therefore \lnot q\\
\end{array}
\]

\vspace{10pt}

\textbf{T18} 证明:$n^2$ 是偶数当且仅当 n 是偶数。

设 p 表示 n 为偶数,q表示 $n^2$ 为偶数。

证$p \rightarrow q$:\\
不妨令 n = 2k, 则$n^2 = 4k^2$,因此$n^2$为偶数。$p \rightarrow q$成立。

证$q \rightarrow p$:\\
反证法:不妨假设n为奇数,且$n^2$为偶数,即 n = 2k + 1.\\
所以$n^2 = (2k + 1)^2 = 4k^2 + 4k + 1$\\
$4k^2 + 4k$为偶数,故$n^2$为奇数,与假设矛盾,故假设不成立。$q \rightarrow p$成立。

综上所述,$n^2$ 是偶数当且仅当$n$是偶数。

\vspace{10pt}

\textbf{T20} 设 $A$ 和 $B$ 是全集 $U$ 的子集,证明$A \subseteq B$ 当且仅当 $\overline{B} \subseteq \overline{A}$

证$(A \subseteq B) \rightarrow (\overline{B} \subseteq \overline{A})$:\\
对于$\forall x \in \overline{B}$,即$x \notin B$。因为$A \subseteq B$,所以$x \notin A$,即$x \in \overline{A}$。 成立。

证$(\overline{B} \subseteq \overline{A}) \rightarrow (A \subseteq B)$:\\
对于$\forall x \in A$,即$x \notin \overline{A}$。因为$\overline{B} \subseteq \overline{A}$,所以$x \notin \overline{B}$,即$x \in B$。成立。

综上所述:$A \subseteq B$ 当且仅当 $\overline{B} \subseteq \overline{A}$。

\vspace{10pt}

\textbf{T22} 证明:$k$为奇数是$k^3$为奇数的一个充要条件。

证$k$为奇数$\rightarrow k^3$ 为奇数\\
令$k = 2n + 1$,所以$k^3 = (2n + 1)^3 = 8n^3 + 12n^2 + 6n + 1$。此时$k^3$为奇数,成立。

证$k^3$为奇数$\rightarrow k$为奇数\\
反证,假设k为偶数,$k^3$为奇数。不妨令 $k = 2n$, 所以$k^3 = 8n^3$ 为偶数,与假设矛盾。故假设不成立,原命题成立。

综上所述:$k$为奇数是$k^3$为奇数的一个充要条件。

\vspace{10pt}

\textbf{T23} 证明或证伪:$n^2 + 41n + 41$对每个整数n都是一个素数。

取$n = 41$,此时$n^2 + 41n + 41 = 2 \times 41 + 41 \times 41 + 41$,显然能被41整除,命题不成立。

\vspace{10pt}

\textbf{T24} 证明或证伪:任何5个连续整数之和能被5整除。

不妨设五个整数分别为$n, n + 1, n + 2, n + 3, n + 4$,和为:$5n + 10$能被5整除,命题成立。

\vspace{10pt}

\textbf{T30} 

论证大致正确,当其中可以强调$b \neq 0$,使论证更为严谨。

\section{\textbf{P79-83}}

\textbf{T20}

(a) 当 $P(n)$ 成立时,即$2 | (2n - 1)$,因为$2 | 2$,所以 $2 | (2n - 1 + 2)$,即$2 | (2n + 1)$,所以$P(k + 1)$也成立。
当 $P(n)$ 不成立时,同理 $P(n + 1)$也不成立。

$P(n) \rightarrow P(n + 1)$的真值表为:\[
\begin{array}{|c|c|c|}
    \hline
    P(n) & P(n + 1) & P(n) \rightarrow P(n + 1)\\
    \hline
    F & F & T\\
    T & T & T\\
    \hline
\end{array}
\]
因此$P(n) \rightarrow P(n + 1)$为重言式。

(b) 对于任意的整数$n$, $2n - 1$为奇数,不能被2整除。因此$P(n)$对于任意整数n不为真。

(c) 不矛盾。分题(a)说明的是$P(n)$与$P(n + 1)$的真值情况一致,说明可以从$P(n)$的真值情况推出$P(n + 1)$
那么数学归纳法的原理也是近似,先找出一两个例子,然后利用递推关系推出剩下的所有情况。因此并不矛盾。\\

\vspace{10pt}

\textbf{T22}

论证时指举出了$P(0)$一个例子,但是后续$2^k$以及$2^{k-1}$都使用了这个特例。\\

\vspace{10pt}

\textbf{T26} 证明:设 $A$和$B$是方阵,如果$AB = AB$,那么$(AB)^n = A^nB^n$ 对 $n \geqslant 1$ 成立。

当$n = 1$时,$AB=BA$已由条件给出,成立。

假设$n = k$时成立,证明$n = k + 1$时也成立:
\begin{align*}
    (AB)^{k + 1} &= (AB)^k AB\\
    &=A^kB^kAB\\
    &=A^kB^{k-1}(BA)B\\
    &=A^kB^{k-1}AB^2\\
    &\ \, \vdots\\
    &=A^{k+1}B^{k+1}\\
\end{align*}

证毕。\\

\vspace{10pt}

\textbf{T28} 证明:每个比27大的整数都能写成$5a+8b$,其中$a, b \in \textbf{N}$

有:$28 = 5 \times 4 + 8 \times 1, \quad 29 = 5 \times 1 + 8 \times 3, \quad 30 = 5 \times 6 + 8 \times 0$,\\ 
$31 = 5 \times 3 + 8 \times 2, \quad 32 = 5 \times 0 + 8 \times 4$.
根据数学归纳法,目前存在五个连续的大于27的整数满足条件。
且P(n)成立时。P(n + 5)显然成立。\\
证毕。

\vspace{10pt}

\textbf{T30} 证明:如果$GCD(a, b) = 1$, 那么对所有$n \geq 1, GCD(a^n, b^n) = 1$.

因为$GCD(a, b) = 1$,即对$a, b$ 分别进行质因数分解时,只有1 这一个共同的公因数。那么当$a, b$分别变为n次方后,
分解出来的质因数与一开始的等同,因此仍然只会存在1这一个唯一的公因数。因此如果$GCD(a, b) = 1$, 那么对所有$n \geq 1, GCD(a^n, b^n) = 1$。

\vspace{10pt}

\textbf{T34}

当循环尚未进行时(第0次循环),$W_0 = Y, Z_0 = X$,有$(Y \times W_0) + Z_0 = X + Y^2$。

假设进入了第n次循环,不变量成立。即:$(Y \times W_n) + Z_n = X + Y^2$成立。

那么第n + 1次时,$W_{n + 1} = W_n - 1, Z_{n + 1} = Z_n + Y$\\
因此
\begin{align*}
    (Y \times W_{n + 1}) + Z_{n + 1} &= (Y \times (W_n - 1)) + (Z_n + Y)\\
    &= (Y \times W_n) + Z_n\\
    &= X + Y^2\\
\end{align*}

因此$(Y \times W) + Z = X + Y^2$在循环过程中始终成立。

在循环结束时,$W = 0$,因此$Z = X + Y^2$。

\vspace{10pt}

\textbf{T36}

当循环尚未进行时(第0次循环),$R_0 = 1, K_0 = 2M$,有$R \times N^K = 1 \times N^{2M} = N^{2M}$。

假设进入了第n次循环,不变量成立。即:$R_n \times N^{K_n} = N^{2M}$。

那么第n + 1次时,$R_{n + 1} = R_n \times N, K_{n + 1} = K_n - 1$\\
因此
\begin{align*}
    R_{n + 1} \times N^{K_{n+1}} &= (R_n \times N) \times (N^{K_n - 1})\\
    &= R_n \times N^{K_n}\\
    &=N^{2M}\\
\end{align*}

因此$R \times N^K = N^{2M}$在循环过程中始终成立。

在循环结束时,$R = N^{2M}, K = 0$, 因此$R = N^{2M}$。

\section{P86-87}

\textbf{T20}

由 $g_1 = 1, g_2 = 3, g_n = g_{n - 1} + g_{n - 2}$,因此$g_3 = 4, g_4 = 7, g_5 = 11, g_6 = 18, g_7 = 29, g_8 = 37, g_9 = 66, g_{10} = 103$.
因为$g_2 + g_4 = g_5 - 1 \quad g_2 + g_4 + g_6 = g_7 - 1 \quad g_2 + g_4 +g_6 + g_8 = g_9 - 1$。提出猜想:$\sum_{i = 1}^{n}g_{2i} = g_{2n + 1} - 1$

\vspace{10pt}

\textbf{T21}
因为$g_1 + g_2 + g_3 + g_4 = g_6 - 3 \quad \sum_{i = 1}^{5}g_i = g_7 - 3$,提出假设:$\sum_{i = 1}^{n}g_i = g_{i + 2} - 3$。

\vspace{10pt}

\textbf{T22}
因为$g_1 + g_3 = g_4 - 2\quad g_1 + g_3 + g_5 = g6 - 2 \quad g_1 + g_3 + g_5 + g_7 = g_8 - 2$,提出猜想:
$\sum_{i = 1}^{n}g_{2i - 1} = g_{2n} - 2$

\vspace{10pt}

\textbf{T23}
验证:$g_2 + g_4 + \dots + g_{2n} = g_{2n + 1} - 1$\\
运用数学归纳法,当n = 1时,$g_2 = g3 - 1$成立。\\
假设当$n = k$时,$\sum_{i = 1}^{n}g_{2i} = g_{2n + 1} - 1$成立,即$\sum_{i = 1}^{k}g_{2i} = g_{2k + 1} - 1$。\\
当$n = k + 1$时,
\begin{align*}
    \sum_{i = 1}^{k + 1}g_{2i} &= \sum_{i = 1}^{k}g_{2i} + g_{2k + 2}\\
    &= g_{2k + 1} - 1 + g_{2k + 2}\\
    &= g_{2k + 3} - 1\\
    &= g_{2(k + 1) + 1} - 1\\
\end{align*}
经检验,猜想成立。

\vspace{10pt}

\textbf{T24}
有T23可知,对于偶数项有:$\sum_{i = 1}^{n}g_{2i} = g_{2n + 1} - 1$,
要检验$\sum_{i = 1}^{n}g_i = g_{i + 2} - 3$,即证明对于奇数项,有$\sum_{i = 1}^{n}g_{2i - 1} = g_{2n} - 2$。
下面验证奇数项的结论:$\sum_{i = 1}^{n}g_{2i - 1} = g_{2n} - 2$。
运用数学归纳法,当$n = 1$时,$g_1 = g_2 - 2$成立。\\
假设当$n = k$时,$\sum_{i = 1}^{n}g_{2i - 1} = g_{2n} - 2$成立,即$\sum_{i = 1}^{k}g_{2i - 1} = g_{2k} - 2$\\
当$n = k + 1$时,
\begin{align*}
    \sum_{i = 1}^{k + 1}g_{2i - 1} &= \sum_{i = 1}^{k}g_{2i - 1} + g_{2k + 1}\\
    &= g_{2k} - 2 + g_{2k + 1}\\
    &= g_{2k + 1} - 2\\
\end{align*}
满足猜想形式,猜想成立。

\vspace{10pt}

\textbf{T25}
运用数学归纳法,当$n = 1$时,$g_1 = g_2 - 2$成立。\\
假设当$n = k$时,$\sum_{i = 1}^{n}g_{2i - 1} = g_{2n} - 2$成立,即$\sum_{i = 1}^{k}g_{2i - 1} = g_{2k} - 2$\\
当$n = k + 1$时,
\begin{align*}
    \sum_{i = 1}^{k + 1}g_{2i - 1} &= \sum_{i = 1}^{k}g_{2i - 1} + g_{2k + 1}\\
    &= g_{2k} - 2 + g_{2k + 1}\\
    &= g_{2k + 1} - 2\\
\end{align*}
满足猜想形式,猜想成立。

\end{document}