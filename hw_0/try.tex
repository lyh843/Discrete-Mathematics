\documentclass[12pt, letterpaper]{article}
\usepackage{graphicx} %图片库
\usepackage{amsmath}  %数学库
\usepackage{CJKutf8}  %中文库
\usepackage{algorithm}
\usepackage{algpseudocode} % 或者 algorithmicx
\graphicspath{{image/}} %图片所处文件夹
\title{My first LaTeX document}
\author{Hubert Fansworth\thanks{Funded by the Overleaf team.}}
\date{\today}
\begin{document}
\maketitle

\tableofcontents

We have now added a title, author and date to our first \LaTeX{} document!

Some of the \textbf{greatest}
discoveries in \underline{science}
were made by \textbf{\textit{accident}}

Some of the greatest \emph{discoveries} in science
were made by accident.

\textit{Some of the greatest \emph{discoveries}
in science were made by accident.}

\textbf{Some of the greatest \emph{discoveries}
in science were made by accident.}

\begin{figure}[h]
    \centering
    \includegraphics[width=0.1\textwidth]{1325.jpg}
    \caption{A nice plot}
    \label{fig:meshl}
\end{figure}

As you can see in figure \ref{fig:meshl}, the function grows near the
origin. This example is on page \pageref{fig:meshl}.

\begin{itemize}
    \item The individual entries are indicated with a black dot, a so-called
bullet.
    \item The text in the entries may be of any length.
\end{itemize}

\begin{enumerate}
    \item This is the first entry in our list.
    \item The list numbers increase with each entry we add.
\end{enumerate}

In physics, the mass-energy equivalence is stated
by the equation $E=mc^2$, discovered in 1905 by Albert Einstein.

\begin{math}
    E=mc^2
\end{math} is typeset in a pararaph using inline math mode---as is 
$E=mc^2$, and so too is \(E=mc^2\).

The mass-energy equivalence is described by the famous equation
\[ E=mc^2 \] discovered in 1905 by Albert Einstein.

In natural units ($c = 1$), the formula expresses the identity
\begin{equation}
E=m
\end{equation}

Subscripts in math mode are written as $a_b$ and superscripts are written as $a^b$. These can be combined and nested to write expressions such as

\[ T^{i_1 i_2 \dots i_p}_{j_1 j_2 \dots j_q} = T(x^{i_1},\dots,x^{i_p},e_{j_1},\dots,e_{j_q}) \]

We write integrals using $\int$ and fractions using $\frac{a}{b}$ $\dfrac{a}{b}$. Limits are placed on integrals using superscripts and subscripts:

\[ \int^1_0 \frac{dx}{e^x} =  \frac{e-1}{e} \]
\[ \int^1_0 \dfrac{dx}{e^x} =  \dfrac{e-1}{e} \]

Lower case Greek letters are written as $\omega$ $\delta$ etc. while upper case Greek letters are written as $\Omega$ $\Delta$.

Mathematical operators are prefixed with a backslash as $\sin(\beta)$, $\cos(\alpha)$, $\log(x)$ etc.

\section{First example}

The well-known Pythagorean theorem \(x^2 + y^2 = z^2\) was proved to be invalid for other exponents, meaning the next equation has no integer solutions for \(n>2\):

\[ x^n + y^n = z^n \]

\section{Second example}

This is a simple math expression \(\sqrt{x^2+1}\) inside text. 
And this is also the same: 
\begin{math}
\sqrt{x^2+1}
\end{math}
but by using another command.

This is a simple math expression without numbering
\[\sqrt{x^2+1}\] 
separated from text.

This is also the same:
\begin{displaymath}
\sqrt{x^2+1}
\end{displaymath}

\ldots and this:
\begin{equation*}
\sqrt{x^2+1}
\end{equation*}

\begin{abstract} %摘要
This is a simple paragraph at the beginning of the
document. A brief introduction about the main subject.
\end{abstract}

After our abstract we can begin the first paragraph, then press ''enter''
twice to start the second one.

This line will start a second paragraph.

I will strt the third paragraph and then add \\ a manual line break which
causes this text to start on a new line but remains part of the same
paragraph. Alternatively, I can use the \verb|\newline|\newline command to
start a new line, which is also part of the same paragraph.

\begin{center}
\begin{tabular}{|c|c|c|}
    \hline
    cell1 & cell2 & cell3 \\
    \hline
    cell4 & cell5 & cell6 \\
    \hline
    cell7 & cell8 & cell9 \\
    \hline
\end{tabular}
\end{center}

\begin{CJK}{UTF8}{gkai}
    你好发
\end{CJK}

\begin{math}
    \alpha \beta \gamma \rho \sigma \delta \epsilon \\
    \times \otimes \oplus \cup \cap \\
    < > \subset \supset \subseteq \supseteq \\
    \int \oint \sum \prod \\
    a_{n_i}  \coprod_{i=1}^n
\end{math}
\[\sum_{i=1}^{\infty}\frac{1}{n^s} = \prod_p\frac{1}{1 - p ^ {-s}}\]

\begin{math}
    \cos \csc \exp \ker \limsup \min \sinh \arcsin \\
    \ln
\end{math}





\begin{algorithm}
\caption{欧几里得算法}
\begin{algorithmic}[1]
\Procedure{Euclid}{$a,b$}\Comment{计算a和b的最大公约数}
\State $r \gets a \bmod b$
\While{$r \neq 0$}\Comment{当余数不为0时}
\State $a \gets b$
\State $b \gets r$
\State $r \gets a \bmod b$
\EndWhile
\State \textbf{return} $b$\Comment{最大公约数是b}
\EndProcedure
\end{algorithmic}
\end{algorithm}

\end{document}