\documentclass[12pt, a4paper]{ctexart}
\usepackage{amsmath,amssymb} % 添加 amssymb 以支持更多符号
\usepackage{bussproofs}
\title{离散数学作业\_0}
\author{李云浩 241880324}
\date{\today}
\begin{document}
\maketitle
\tableofcontents

\section{逻辑和证明}
\subsection{符号}

% 使用 itemize 列表整理符号,用数学模式包裹符号
\begin{itemize}
    \item 逻辑运算符:
    \begin{itemize}
        \item 与:$\wedge$ 或 $\land$(\verb|\wedge| 或 \verb|\land|)
        \item 或:$\vee$ 或 $\lor$(\verb|\vee| 或 \verb|\lor|)
        \item 非:$\lnot$(\verb|\lnot|)
    \end{itemize}
    
    \item 关系符号:
    \begin{itemize}
        \item 小于等于:$\leq$ 或 $\leqslant$(\verb|\leq| 或 \verb|\leqslant|)
        \item 大于等于:$\geq$ 或 $\geqslant$(\verb|\geq| 或 \verb|\geqslant|)
        \item 子集:$\subset$, $\subseteq$(\verb|\subset|, \verb|\subseteq|)
        \item 超集:$\supset$, $\supseteq$(\verb|\supset|, \verb|\supseteq|)
        \item 属于:$\in$(\verb|\in|)
        \item 不属于: $\notin$(\verb|\notin|)
        \item 逻辑等价:$\equiv$(\verb|\equiv|)
    \end{itemize}
    
    \item 箭头符号:
    $\leftarrow$(\verb|\leftarrow|),
    $\rightarrow$(\verb|\rightarrow|),
    $\leftrightarrow$(\verb|\leftrightarrow|)

    \item 希腊字母:
    $\alpha$(\verb|\alpha|),
    $\beta$(\verb|\beta|)

    \item 量词:
    $\forall$(全称量词,\verb|\forall|),
    $\exists$(存在量词,\verb|\exists|),
    $\nexists$(不存在,\verb|\nexists|)
    
    \item 因果符号:
    \begin{itemize}
        \item 因为:$\because$(\verb|\because|)
        \item 所以:$\therefore$(\verb|\therefore|)
    \end{itemize}
\end{itemize}

\subsection{题目示例}
\subsection*{题目}
\noindent
构建论证证明:通过第一个考试的某个人没有读过这本书\newline
前提:这个版有个学生没有读过这本书;这个班每个人都通过了第一次考试\newline
$x$的论域是全体学生,$C(x):$ $x$在这个班; $B(x):$ $x$读过这本书;$P(x):$ $x$通过第一次考试
\subsection*{分析}
\noindent
即证:
\[
\begin{array}{c}
    \exists x(C(x) \land \lnot B(x)) \\
    \forall x(C(x) \rightarrow P(x)) \\
    \hline 
    \therefore \exists x(P(x) \land \lnot B(x)) \\
\end{array}
\]
\subsection*{解题过程}
\[
\begin{array}{rl @{\quad} l}
    1. & \exists x(C(x) \wedge \neg B(x)) & \text{前提} \\
    2. & C(a) \wedge \neg B(a) & \text{EI} \\
    3. & C(a) & \text{对 2 应用化简率} \\
    4. & \forall x(C(x) \rightarrow P(x)) & \text{前提} \\
    5. & C(a) \rightarrow P(a) & \text{UI} \\
    6. & P(a) & \text{3 和 5 应用假言推理} \\
    7. & \neg B(a) & \text{对 2 应用化简率} \\
    8. & P(a) \wedge \neg B(a) & \text{对 6 和 7 应用合取律} \\
    \hline
    9. & \therefore \exists x(P(x) \wedge \neg B(x)) & \text{EG} \\
\end{array}
\]

\end{document}