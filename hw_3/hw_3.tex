\documentclass{article}
\usepackage{amsthm}
\usepackage{graphicx}
\usepackage{ctex}
\usepackage{amsmath}
\usepackage{amssymb} % 添加 amssymb 以支持更多符号
\usepackage{amsfonts}
\usepackage{tikz} % 文氏图
\usetikzlibrary{shapes.geometric, arrows.meta, positioning}
\title{离散数学作业\_3}
\author{李云浩 241880324}
\date{\today}
\begin{document}
\maketitle
\section{1.3}
\subsection{T23}
A 无限集、不可数。\quad B 有限集、可数 \quad C 无限集、可数 \quad 
D 有限、可数 \quad E 有限、可数
\subsection{T24}
A 无限、可数 \quad B 有限、可数 \quad C 无限、可数 \quad D 无限、不可数 \quad
E 无限、不可数
\section{3.1}
\subsection{T25}
$\frac{A_n^n}{n} = (n - 1)!$
\subsection{T26}
先把n个人排成一列,共有$A_n^n$种排序方法,但由于最后是围绕称一个圆桌。
因此相当于把这一列首尾相连,此后谁是首无需在意。因此在次序不变时,n个人轮流为
首的n种情况是等价的。因此要对原来的结果除以n。故答案为:$\frac{A_n^n}{n}$
\subsection{T29}
证明:$n \times _{n-1}P_{n-1} = _nP_n$\\
由于:$_nP_n = \frac{n!}{(n - n)!} = n!, \quad _{n-1}P_{n-1} = \frac{(n-1)!}{[n - 1 - (n - 1)]} = (n - 1)!$\\
所以:$n \times _{n-1}P_{n-1} = n \times (n - 1)! = n! = _nP_n$.证毕。

\subsection{T34}
末位0的个数取决于对$n!$的项进行质因数分解后可以组成多少个$2 \times 5$的搭配。因为2的数量显然要比5多,
因此末尾0的个数只取决于$n!$的项中的因数5的数目。其中,25中有两个因数5,125中有三个因数5,……
因此对于$5^k \leq n < 5^{k+1}$,$n!$中某位0的个数为:$\left[\frac{n}{5}\right] + \left[\frac{n}{25}\right]
+ \dots + \left[\frac{n}{5^k}\right]$。其中$\left[x\right]$表示不大于x的最大整数。

\section{3.2}
\subsection{T19}
证明:$_{n+1}C_{r} = _nC_{r-1} + _nC_r$
因为:$_nC_r = \frac{n!}{r!(n-r)!}$,\\
所以:$_{n+1}C_r = \frac{(n + 1)!}{r!(n+1 - r)!} \quad _nC_{r-1} = \frac{n!}{(r - 1)!(n - r +  1)!}$,
因此:
\begin{align*}
    _nC_{r-1} + _nC_{r} &= \frac{n!}{(r - 1)!(n - r +  1)!} + \frac{n!}{r!(n-r)!}\\
    &= \frac{rn!}{(n - r + 1)r!(n-r)!} + \frac{n!}{r!(n-r)!}\\
    &= \left(1 + \frac{r}{n-r+1}\right)\frac{n!}{r!(n-r)!}\\
    &= \frac{(n + 1)n!}{(n - r + 1)(n-r)!}\\
    &= \frac{(n + 1)!}{r!(n+1 - r)!}\\
    &= _{n+1}C_r
\end{align*}证毕。
\subsection{T23}
(a)每次有两种可能,故n次的记录序列为$2^n$。\\
(b)恰好有3个反面,即在n个记录里抽3个记录,令其为反面,故为:$C_n^3$。\\
(c)恰好包含k个正面,即在n个记录里抽k个记录,令其为正面,故为:$C_n^k$。
\subsection{T27}
(a)后继第一行:$1 \ 5 \ 10 \ 10 \ 5 \ 1$,后继第二行:$1 \ 6 \ 15 \ 20 \ 15 \ 6 \ 1$,后继第三行:
$1 \ 7 \ 21 \ 35 \ 35 \ 21 \ 7 \ 1$。
(b)设由第n - 1行得到第n行,满足下列规则:$a_{(n,1)} = a_{(n, n)} = 1,\quad a_{(n, k)} = a_{(n-1, k - 1)} + a_{(n-1, k)}$
\subsection{T32}
猜想:$2^{n - 1}$。\\
证明:因为Pascal三角形满足下列递推关系:$$a_{(n,1)} = a_{(n, n)} = 1,\quad a_{(n, k)} = a_{(n-1, k - 1)} + a_{(n-1, k)}$$
其中Pascal三角形的起始条件为:$a_{2, 1} = a_{2, 2} = 1$。
因为对于$C_n$存在同样的递推关系:$$C_n^0 = C_n^n = 1, \quad C_n^r = C_{n-1}^r + C_{n - 1}^{r - 1}$$
其中$C_n$的起始条件为:$C_1^0 = C_1^1 = 1$。
可以观察到二者的递推关系以及起始条件等价,$C_n^k$与$a_{n + 1, k + 1}$也等价。
因此$\sum_{i = 0}^{n}C_n^i = \sum_{i = 1}^{n + 1}a_{n + 1, i}$,故$\sum_{i = 1}^{n}a_{n, i} = 
\sum_{i = 0}^{n - 1}C_{n - 1}^i = 2^{n - 1}$。
\section{3.3}
\subsection{T10}
因为前六个数的和为:$1+2+3+4+5+6 = 21$,后六个数的和为:$10+11+12+13+14+15 = 75$。
两者之差为$75 - 21 = 54$,故区间长度为55。任意选6个整数,它们之和一定也落在这个长度为55的区间内。
因为$C_15^6 = 5005$,根据鸽巢原理,把5005个数分配到长度为55的区间内,一定有$\frac{(5005 - 1)}{55} + 1 > 91$。
因此一定有90种方法的使得所有的选择有相同的和。
\subsection{T12}
若题目为不超过$\sqrt{2}$,则对于边长为1的正方形而言,最长的线段为对角线长度为$\sqrt{2}$,不超过$\sqrt{2}$,因此显然满足。

若题目为不超过$\frac{\sqrt{2}}{2}$,可以将大正方形平分为4个边长为$\frac{1}{2}$的小正方形,因为一共有5个点,因此一定有一个
小正方形里面有两个点。因为小正方形内最长的线段为对角线,长度为$\frac{\sqrt{2}}{2}$,因此,该两个点的距离不超过$\frac{\sqrt{2}}{2}$。
证毕。
\subsection{T17}
当n为偶数时,当子集取$\{\frac{n}{2}, \frac{n}{2} + 1, \dots, n\}$时,显然这个子集不存在一个数是另一个数的倍数。
但是当取多一个$\frac{n}{2} - 1$时,那么$2 \times \left(\frac{n}{2} - 1\right)$一定在这个子集之中。因此子集大小为:$\frac{n}{2} + 1$。
当n为奇数时,n - 1为偶数,因此子集取$\frac{n + 1}{2} + 1$。

综上所述:子集应该取$\left[\frac{n + 1}{2}\right] + 1$大小时,才能保证该子集中的一个数是同一个子集的另一个数的倍数。

\subsection{T18}
有可能,当选出的10张为:$1 \sim 10$时,其中任选两张卡片组成的最大的和为19,小于21,获胜。

\subsection{T19}
要使两张牌的和为21,那么组合可以是:$1 + 20, 2 + 19, \dots, 10 + 11$,共有10种搭配,
要不使两张牌的和为21,则选的牌里面不可以有两张牌能组成上面的一个搭配。那么假设每种搭配只选取一张,
那么我只能选取出10张,剩下的两张必定会与一开始选的10张之中的两张成功配对。因此不可能赢得这个游戏。
\subsection{T21}
考虑和数$c_1, c_1 + c_2, \dots, c_1 + c_2 + c_3 + c_4 + c_5 + c_6$,一共有六个数,分别除以6时,
会得到相应的余数,记为:$n_1, n_2, \dots ,n_6$。因为除以6之后的余数可能仅有6种可能。
若$n_1 \sim n_6$中,存在$n_k = 0$,则说明其对应的和数被6整除。若$n_1 \sim n_6$中,不存在$n_k = 0$,
则该六个余数对应只有5种可能,根据鸽巢原理,必有$n_i = n_j, i < j$。因此用第j个和数与第i个和数做差,
其差同样为该六个整数的任意序列的和,并且能被6整除。证毕。
\subsection{T22}
考虑和数$c_1, c_1 + c_2, \dots, c_1 + c_2 + \dots + c_n$,分别记为$m_1, m_2, \dots m_n$,一共有n个数。
分别除以n时会得到相应的余数,记为:$q_1, q_2, \dots, q_n$。因为除以n之后的余数可能仅有n种可能。
若$q_1 \sim q_n$种,存在$q_k = 0$,则说明$m_k$能够被n整除。若$q_1 \sim q_n$中,不存在$q_k = 0$,
则该n个余数对应只有$n - 1$种可能,根据鸽巢原理,必有$n_i = n_j, i < j$。因此用第j个和数与第i个和数做差,
其差为:$c_{i + 1} + c_{i + 2} + \dots + c_j$,为n个整数的任意序列的和,并且能被n整除。证毕。
\subsection{T23}
讨论六个正整数中1的个数。
\begin{itemize}
    \item 有三个及以上个1,那么必有子集\{1, 1, 1\},其和为3。
    \item 有且只有两个1,如果存在2的话,那么有子集, \{1, 2\},其和为3。
如果不存在2的话,那么剩下4个数字最小为4(因为如果含有3的话,那么3自己就是一个和为3的子集),
$1 + 1 + 4 \times 4 = 18 > 13$,不满足。
    \item 有且只有一个1,如果存在2的话,那么有子集, \{1, 2\},其和为3。
如果不存在2的话,那么剩下5个数字最小为4,显然不满足和为13。
    \item 如果一个1也不存在的话,由于和为13,显然需要一个奇数元素,那么这个奇数元素只能是5,
其他5个元素最小为2,$5 + 5 \times 2 = 15 > 13$,不满足。
\end{itemize}
综上所述,6个正整数的任意集合之和若是13,则它一定有一个和为3的子集。
\subsection{T24}
任意一个有理数都可以表示为$\frac{a}{b}$,其中$a, b\in \mathbf{Z}, b > 0$。
对于任意的一个b,其可能的余数的数目为b个。
因此$\frac{a}{b}$利用保留余数的除法,除上b次之后,根据鸽巢原理,必定会有一次余数为0或者两次余数相等。
余数为0时,说明除出来的是一个有限的数。两次余数相等说明会有重复的部分。证毕。

\section{3.4}
\subsection{T34}
(a)$P_a = \frac{5 \times 3}{6^3} = \frac{5}{72}$ \quad
(b)$P_b = \frac{10 + 6 + 3 + 1}{6^3} = \frac{5}{54}$ \quad
(c)$P_c = 1 - \frac{5^3}{6^3} = \frac{91}{216}$\\
(d)$P_d = \frac{3 \times 5 \times 5 + 5^3}{6^3} = \frac{25}{27}$ \quad
(e)$P_e = \frac{5^3}{6^3} = \frac{125}{216}$
\subsection{T37}
当关键字在第一个位置的时候,需要一步;在第二个位置的时候,需要两步;………;在第n个位置的时候,需要n步。
因为关键字在各个位置的概率是相等的,因此平均步数为:$\sum_{k = 1}^{n}\left(\frac{k}{n}\right) = \frac{n + 1}{2}$。
\subsection{T38}
如果不假设关键字在数组里,那么该题就成为了一个条件概率。设事件A为关键字在数组里,事件B为找到字的平均步数。
因此$P(B \mid A) = \frac{n + 1}{2}, \quad P(B \mid \overline{A}) = 0$。
\subsection{T39}
游戏者获胜的情况分别为:$\{(4, 6), (6, 4), (5, 5), (5, 6), (6, 5), (6, 6)\}$,共有6种情况。
游戏期望为:$10 \times \frac{3}{36} + 11 \times \frac{2}{36} + 12 \times \frac{1}{36} - 3 = -\frac{11}{9}$。
\subsection{T40}
要是一个游戏公平,那么总期望应该为0。设花费为x,那么:
\[10 \times \frac{3}{36} + 11 \times \frac{2}{36} + 12 \times \frac{1}{36} - x = 0\]
解得:x = $\frac{16}{9}$。
\subsection{T41}
(a)满足号数为奇数且不是黑色的纸牌一共有8张,因此:\\
$\frac{C_{8}^2}{C_{52}^2} = \frac{14}{663}$。

(b)满足号数为奇数且不是黑色的纸牌一共有8张,但是不是黑色的纸牌有26张。
因此:假设有且只有一张是奇数,$P_1 = 2 \times \frac{C_{8}^1}{C_{52}^1} \times \frac{C_{26}^1}{C_{52}^1} = 
\frac{2}{13}$。假设两张都是奇数,由(a)可知,$P_2 = \frac{14}{663}$。因此总概率为:$P_1 + P_2 = \frac{116}{663}$
\section{3.5}
\subsection{T14}
\begin{align*}
    c_n &= c_{n - 1} + n\\
    &= c_{n-2} + n + (n - 1)\\
    &\ \ \vdots\\
    &= c_1 + n + (n - 1) + \dots + 2
\end{align*}
因为$c_1 = 4$,因此$c_n = 3 + \frac{(n + 1)n}{2}$。
\subsection{T18}
因为$a_n = 4a_{n-1} + 5 a_{n -2}$,得特征方程:$r^2 - 4r - 5 = 0$。
解得:$r_1 = 5, r_2 = -1$。设$a_n = \alpha_1 5^n + \alpha_2 (-1)^n$。
因为$a_1 = 2, a_2 = 6$,带入解得:$\alpha_1 = \frac{4}{15}, \alpha_2 = -\frac{2}{3}$。
所以:$a_n = \frac{4}{15}5^n - \frac{2}{3}(-1)^n$。
\subsection{T26}
(a) $r_n = r_{n - 1} + n$。

(b)\begin{align*}
    r_n &= r_{n - 1} + n\\
    &= r_{n - 2} + n + (n - 1)\\
    &\ \ \vdots\\
    &= r_1 + n + (n - 1) + \dots + 2 
\end{align*}
因为一条直线的时候,把平面分成了两个部分,即$r_1 = 2$。
因此$r_n = \frac{(n + 1)n}{2} + 1$。
\subsection{T28}
因为$x^2 - r_1x - r_2 = 0$有唯一单根s,因此$r_1^2 + 4r_2 = 0, s^2 - r_1s - r_2 = 0
a_n = us^n + vns^n$,尝试推导出$a_n = r_1a_{n - 1} + r_2a_{n-2}$。
\begin{align*}
    a_n &= us^n + vns^n\\
    &= us^{n - 2}s^2 + vns^{n - 2}s^2\\
    &= us^{n - 2}(r_1s + r_2) + vns^{n - 2}(r_1s + r_2)\\
    &= r_1us^{n - 1} + r_2us^{n - 2} + r_1vns^{n - 1} + r_2vns^{n - 2}\\
    &= r_1(us^{n - 1} + vns^{n - 1}) + r_2(us^{n - 2} + vns^{n - 2})\\
    &= r_1a_{n - 1} + r_2a_{n - 2}
\end{align*}
证毕。
\subsection{T34}
(a)三阶。
(b)由:$a_n = 7a_{n - 2} + 6a_{n - 3}$,得特征方程:$r^3 - 7r - 6 = 0$。
解得:$r_1 = -1, r_2 = 3, r_3 = -2$。
因此:$a_n = \alpha_1 (-1)^n + \alpha_2 (3)^n + \alpha_3 (-2)^n$。
因为:$a_1 = 3, a_2 = 6, a_3 = 10$。解得:$\alpha_1 = -\frac{7}{2}, \alpha_2 = \frac{17}{30}
, \alpha_3 = \frac{11}{10}$。
故递归关系为:$a_n = -\frac{7}{2}(-1)^n + \frac{17}{30}(3)^n + \frac{11}{10}(-2)^n$。
\subsection{T36}
因为:$b_1 = 1, b_2 = 3, b_n = b_{n - 1} + 2b_{n - 2}$,因此$b_3 = 7 < (\frac{5}{2})^3 = 15.625$。
假设对$n = k - 1, n = k - 2$时,有$b_{k - 1} < (\frac{5}{2})^{k - 1}, b_{k - 2} < (\frac{5}{2})^{k - 2}$。
下证:对于$n = k$,有$b_k < (\frac{5}{2})^k$成立。
\begin{align*}
    b_k &= b_{k - 1} + 2 b_{k - 2}\\
    &< (\frac{5}{2})^{k - 1} + 2 (\frac{5}{2})^{k - 2}\\
    &= \frac{2}{5}(\frac{5}{2})^k + \frac{8}{25}(\frac{5}{2})^k\\
    &= \frac{18}{25}(\frac{5}{2})^k\\
    &< (\frac{5}{2})^k
\end{align*}
证毕。
\end{document}